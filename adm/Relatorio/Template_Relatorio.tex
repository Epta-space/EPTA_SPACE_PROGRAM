%%%%%%%%%%%%%%%%%%%%%%%%%%%%%%%%%%%%%%%%%
% Template formal para se utilizar na equipe de propulsão e tecnologia aeroespacial (EPTA)
% LaTeX Template
% Version 1.0 (16/07/2019)
%
% This template was downloaded from:
%
% Original author:
% Peter Wilson (herries.press@earthlink.net) with modifications by:
% Vel (vel@latextemplates.com)
% Vel (feliperibeiro.ufu@gmail.com)
% Vel (mairaf_miranda@hotmail.com)
%
% License:
% CC BY-NC-SA 3.0 (http://creativecommons.org/licenses/by-nc-sa/3.0/)
%
% Conselhos para se lidar com este template:
% - Evitar alterar códigos referentes à formatação, com exceção de se copiar e colar
%          o código para exibição de imagens, tabelas e equações.
% - Procurar sempre manter as imagens utilizadas no documento em uma pasta dedicada,
%          de forma que, ao se referenciar a imagem, se referencie o caminho para 
%          ela, como feito com a "imagem_de_missao".
%
%
%%%%%%%%%%%%%%%%%%%%%%%%%%%%%%%%%%%%%%%%%

%----------------------------------------------------------------------------------------
%	Pacotes utilizados.(evitar mudar, se for necessária alguma mudança, comunicar gerencia)
%----------------------------------------------------------------------------------------

\documentclass[a4paper, 12pt,  oneside,a4paper,english,french,spanish,brazil]{book}     % Para folha A4, default 11pt tamanho
\usepackage[utf8]{inputenc}                                               % Acentos
\usepackage[brazil]{babel}                                                % Biblioteca para português
\usepackage{amssymb}                                                      % Biblioteca para símbolos
\usepackage{graphicx}                                                     % Biblioteca para figuras
\usepackage{tikz}                                                         % Pacote gráfico
\usepackage{amsmath}                                                      % Biblioteca para símbolos
\usepackage[T1]{fontenc}                                                  % encode para acentos 
\usepackage{fouriernc}                                                    % New Century Schoolbook font
\usepackage{xcolor}                                                       % Controle de cores
\usepackage[many]{tcolorbox}                                              % Controle de cores
\usepackage{amsthm}                                                       % Suporte matemático ams
\usepackage{physics}                                                      % Suporte em notações
\usepackage{gensymb}                                                      % Suporte em notação
\usepackage{amsfonts}                                                     % Suporte em fontes ams
\usepackage[left=1.00cm, right=1.00cm, top=3cm, bottom=2.50cm]{geometry}  % Parâmetros geométricos
\usepackage{caption}                                                      % Legenda para imagens
\usepackage{subcaption}                                                   % Legenda para figuras múltiplas
\usepackage{capt-of}                                                      % Suporte com legendas
\usepackage{float}                                                        % Suporte para maiores floats  
\usepackage[figurename=Fig.]{caption}                                     % Suporte 2 títulos legendas
\usepackage{tabularx}                                                     % suporte com tabelas
\usepackage{ragged2e}                                                     % suporte localização texto
\usepackage{mathrsfs}                                                     % Fontes para equações
\usepackage{makecell}                                                     % Formatação de tabelas
\usepackage{textpos}                                                      % Facilita posicionamento de forma absoluta
\usepackage[hyphens]{url}                                                 % Possibilita uso de hífen em links
\usepackage[makeroom]{cancel}                                             % Desconsidera limites para retas
\usepackage{empheq}                                                       % Possibilita emoldurar fórmulas
\usepackage[colorlinks=true,linkcolor=blue,urlcolor=blue,bookmarksopen=true, bookmarksnumbered, pdfencoding=auto]{hyperref}                                               % links clicáveis em PDF
\usepackage{fancyhdr}                                                     % Ajuda na criação de footers e headers
\usepackage{siunitx}                                                      % General suporte
\usepackage{enumitem}                                                     % Controle sobre layout
\usepackage{cancel}                                                       % Riscar coisas
\usepackage{bookmark}                                                     % Hiperlink em PDF
\usepackage{versions}                                                     % Devido às versões
\usepackage[pages=some,scale=1,angle=0,opacity=1]{background}             % Imagem de Background
\usetikzlibrary{decorations.pathmorphing}                                 % Permite utilização direta fonte abaixo
\usetikzlibrary{shapes.callouts,positioning}                              % Balões   
\usetikzlibrary{shapes.geometric,arrows,shadows}                          % flechas
\usepackage{frcursive}                                                    % Fonte cursiva
\usepackage{multirow}                                                     % Suporte tabelas

\setlength\parindent{3ex}                                                 % Parâmetro geométrico
\setlength{\parskip}{0.25em}                                              % Parâmetro geométrico
\setlength{\fboxsep}{10pt}                                                % Espaço ao redor dos quadros
\definecolor{zelena}{RGB}{0,100,0}                                        % Define cores em estruturas
\tcbuselibrary{theorems}                                                  % Cor nas caixas                           
\tcbuselibrary{breakable}                                                 % Particiona coisas
\pagestyle{fancy}  
\fancyhead[RO , RE]{\footnotesize 
	\begin{minipage}[b]{0.435\linewidth}\flushright\leftmark\end{minipage}}  % cabeçalho direito ímpar=título do capítulo
\fancyhead[LO , LE]{\footnotesize 	
	\begin{minipage}[b]{0.435\linewidth}\flushleft\rightmark\end{minipage}} % cabeçalho esquerdo = nome da seção
\fancyfoot[CE,CO]{\thepage}                                               % Nota de pé
\allowdisplaybreaks                                                       % Possibilita alguns comandos de controle
\setcellgapes{5pt}                                                        % Formatação de tabelas
\definecolor{eggshell}{rgb}{0.94, 0.92, 0.84}                             % Definição de cor 



\setlength\headheight{27.05003pt}
\renewcommand\headrule{\hrulefill
	\raisebox{10.1pt}[-10pt][-10pt]{\plogo}\hrulefill}                      % Headers
\newcommand{\plogo}{\fbox{  $EPTA$   }}                                   % sigla equipe 
\newcommand\BackImage[2][scale=1]{\BgThispage\backgroundsetup{
		contents={\includegraphics[#1]{#2}}}}                             % cria a função background
\newtcbox{\caixaeq}[1][]{nobeforeafter,math upper,tcbox raise base,enhanced,
	colframe=black!50!black,colback=eggshell!40!white,arc=4pt,	
	boxrule=1pt,drop fuzzy shadow,#1}                                     % Moldura para fórmulas

\DeclareMathOperator{\tg}{tg}                                             % Declara construção mat. 
\DeclareMathOperator{\cotg}{cotg}                                         % Declara construção mat.

%%%%%%%%%%%%%%%%%%%%%%%%%%%%%%%%%%%%%%%%%
\frontmatter
\date{}
% \thispagestyle{empty}
\date{\today}
\sloppy
\usepackage{eso-pic}
\newcommand\BackgroundPic{%
\put(0,0){%
\parbox[b][\paperheight]{\paperwidth}{%
\vfill
\centering
\includegraphics[width=\paperwidth,height=\paperheight,%
keepaspectratio]{midia/epta_projeto_template}%
\vfill
}}}

%----------------------------------------------------------------------------------------
%    INÍCIO DO DOCUMENTO    (a partir daqui pode editar sem grandes complicações)
%----------------------------------------------------------------------------------------
\begin{document} 

%----------------------------------------------------------------------------------------
%	PÁGINA DE TÍTULO
%----------------------------------------------------------------------------------------

\begin{titlepage} % Suprime cabeçalhos e notas na base da folha.

	\AddToShipoutPicture*{\BackgroundPic}
	\begin{minipage}[h!]{0.36\textwidth}
        \phantom{..}
    \end{minipage} 
    \begin{minipage}[h!]{0.6\textwidth}
	
		\centering % centraliza os textos na página
	
		\scshape % Usa formatação pequena 
	
		%------------------------------------------------
		%	Título
		%------------------------------------------------
	
		\rule{\textwidth}{1.6pt}\vspace*{-\baselineskip}\vspace*{2pt} % Linha horizontal grossa
		\rule{\textwidth}{0.4pt} % Linha horizontal fina
	
		\vspace{0.75\baselineskip} % espaço sobre o título
	
		{\LARGE EPTA Entertainment\\} % Título
	
		\vspace{0.75\baselineskip} % espaço abaixo do título
	
		\rule{\textwidth}{0.4pt}\vspace*{-\baselineskip}\vspace{3.2pt} % Linha horizontal fina
		\rule{\textwidth}{1.6pt} % Linha horizontal grossa
	
		\vspace{2\baselineskip} %espaço em branco abaixo do título
	
		%------------------------------------------------
		%	Subtítulo
		%------------------------------------------------
	
		Neste documento encontra-se o plano de desenvolvimento que se refere à criação da versão $1.0$ do aplicativo EPTA Space Program, desenvolvido pela EPTA Entertainment.% descrição adicional (sub-título)
	
		\vspace*{3\baselineskip} % espaço abaixo do subtítulo
	
		%------------------------------------------------
		%	Autores(s)
		%------------------------------------------------
	
		Desenvolvido por
	
		\vspace{0.8\baselineskip} % Espaço antes de autores
		
        {
        Felipe J. O. Ribeiro \\ 
        Mateus da Silva Fernandes \\ 
        Olavo Caetano Inácio \\
        Pedro Guilherme R. V. de Melo
        } % Lista de autores (ordem alfabética)
        
		
        \vspace{0.8\baselineskip} % Espaço após os autores
	
		\textit{Universidade Federal de \\ Uberlândia} % Entidades envolvidas
	
		\vspace{12.8\baselineskip} % Espaço (diminuir se houver mais autores)
	
		{\large Equipe de Propulsão e Tecnologia Aeroespacial} % equipe
		
		\vspace{0.3\baselineskip} % Espaço embaixo da logo
		
		EPTA Space Program - Versão 1.0 % Núcleo análogo
		
		\vspace{0.3\baselineskip} % Espaço embaixo da logo
		
		\today % Data de compilação
		
	\end{minipage}



\end{titlepage}

%----------------------------------------------------------------------------------------
%   Abstract...     (Um resumo do que será tratado em todo o texto)
%----------------------------------------------------------------------------------------
{
	\thispagestyle{empty}     % Sem numeração no abstract
	
	{\LARGE Resumo} % Abstract (grande)
    \vspace{1cm}
	
    Neste documento encontram-se as práticas adotadas no desenvolvimento da versão $1.0$ do aplicativo Epta Space Program. O programa já havia sido lançado na google play em meados de 2017, mas ele foi retirado do ar devido a questões burocráticas (ausência de uma política de privacidade). 

    Nesta etapa de desenvolvimento a equipe atualizará o jogo para a versão mais recente do unity.
    Também revisará o work flow uma vez que na versão de 2017 estava impossível de escalar e inviável de manter devido a más práticas de programação.
    Como resposta a isso, a sub área se propôs a revisar todos os padrões de desenvolvimento de forma a estar conforme com as mais altas recomendações desta indústria.

    As features que espera-se que estejam presentes no jogo na data de estreia são:

    \begin{description}[font=$\bullet$~\normalfont\scshape]
        \item [Arcade] Mecânica base do jogo mantém-se arcade, com rolagem para cima com obstáculos e replay ágil.
        \item [Assests reutilizados] Serão utilizados os assets do antigo lançamento. Novos devem ser feitos conforme necessidade.
        \item [Propaganda] As propagandas tipo banner e tipo Intersticial devem ser implementadas novamente.
        \item [Versão paga] Deve haver uma versão sem propagandas e que é paga. (preço a determinar)
        \item [Tabela Info] Tabela de informações sobre a equipe no menu opções.
        \item [Tela inicial] Desenvolvimento da tela inicial com animações. Que não precise iniciar o jogo pausado.
        \item [Volume] Deve ser possível ajustar o volume.
        \item [Explosão] Som e efeito visual de explosão.
        \item [Salvar] A melhor pontuação do jogador deve ser salva localmente, assim como o volume.
        \item [3 fases] Três etapas distintas de jogo. (céu azul, extratosfera, espaço)
        \item [Alturas reais] As alturas de mudança de fase devem ser consistentes com o mundo real.
        \item [Dificuldade crescente] O sistema de dificuldade deve ser implementado de forma a permitir razoável controle em função de fase e tempo.
        \item [Fase final] A certa altura (a determinar), o player tem a opção de de ir para a lua. (o conteúdo será adicionado posteriormente. Mas por hora o player recebe uma mensagem de congratulações)
    \end{description}

    A metodologia adotada para a criação do APP tem duas etapas importantes: Até o término da versão $1.0$ será adotada uma metodologia em cascata, nesta que é a etapa de pré lançamento, onde se dividirão as features a implementar em uma ordem lógica de execução. 
    Datas para pontos chave no desenvolvimento serão determinadas seguindo o padrão industrial (pre-alpha : $0.5.0$, alpha : $0.6.0$, beta-1 : $0.7.0$, beta-2 : $0.8.0$, ..., release : $1.0.0$), onde cada uma terá um objetivo e uma lista de requerimentos a serem atendidos.
    Em seguida, após lançado, será adotada a metodologia ágil para desenvolvimento contínuo de atualizações e melhoramentos conforme necessário. Esse será chamada de etapa contínua de pós lançamento.

    Será implementado um ciclo de desenvolvimento completamente integrado à ferramente Git de controle de versão e à página GitHub (\url{https://github.com/Epta-space/EPTA_SPACE_PROGRAM}), onde o jogo será salvo em cloud em um repositório privado.
    Assim, essas ferramentas servirão para possibilitar o desenvolvimento em grupo do código fonte, além de também ser uma forma de quantificação da produtividade, uma vez que o controle do número de commits é uma métrica importante no acompanhamento das atividades.
    Mais detalhes podem ser encontrados na descrição das etapas de desenvolvimento do pré lançamento.

    A criação do produto mínimo viável já foi feita na primeira versão do jogo, motivo pelo qual não temos a etapa MVP neste texto. 
    Apesar disso será inserida aqui alguma documentação sobre esse processo nos capítulos iniciais.
    O desenvolvimentos que ocorreram anteriores a esse relatório são o motivo, também, da enumeração das etapas começarem em $0.5$.
    Nos capítulos iniciais também será dada uma detalhada noção do estado atual do aplicativo assim como algumas informações sobre como foi o desenvolvimento até então.
    Tudo que foi feito até o momento foi desenvolvido durante o ano corrido de 2020, com exceção dos assets que vem da versão anterior.

    Dentre as informações do estado atual do aplicativo serão listados os scripts que existem atualmente em formato UML e por escrito, a estrutura de arquivos no projeto unity e a estrutura de Prefabs.
    Estas estruturas também já atendem a padrões de projeto sugeridos online e facilitam o desenvolvimento do controle de versão via Git, além de organizar as informações para facilitar o acesso e aumentar a escalabilidade.

    Próximo ao fim do documento também serão apresentadas algumas propostas de melhoramento na etapa ágil de lançamento, isso é, após a primeira versão do programa ser lançada na Google Play. Isso mostra as possíveis rotas que a subárea tomará no futuro, após o começo da etapa de desenvolvimento contínuo.

	\vspace{1cm}
	{\large Palavras chave:} {\small Desenvolvimento de Software, aplicativo, android, jogo, google play.}

}
%----------------------------------------------------------------------------------------
%	Desenvolvimento...
%----------------------------------------------------------------------------------------

\tableofcontents          % Índice
\thispagestyle{empty}     % Sem numeração na página do índice
\listoffigures            % Lista de figuras 
\thispagestyle{empty}     % Sem numeração na página da lista de imagens
\mainmatter

\chapter{O desenvolvimento até aqui}

    O início deste projeto ocorreu no começo do primeiro semestre de 2018. Um grupo de membros se juntou no desenvolvimento do aplicativo como uma nova proposta de angariação de fundos para a equipe. 
    Observou-se o potencial do mercado de aplicativos global, que cresce consistentemente ao longo dos anos.
    Além disso pensou-se na oportunidade de se aprofundar no desenvolvimento de software e o aprendizado que isso traria.

    Ao longo do ano o aplicativo foi desenvolvido e finalmente lançado em dezembro de 2018. Ele foi baixado em mais de 40 países e mais de 800 vezes, com uma avaliação média de $4.88$ estrelas de $5$. 

    \begin{figure}[h!]
        \centering
        \includegraphics[trim = {0cm 0cm 0cm 0cm}, clip , angle=0, scale=0.55]{midia/users_global.png}
        \caption{Downloads do aplicativo distribuídos no tempo.}
        \label{pika1}
    \end{figure}

    Em julho de 2019 o aplicativo foi retirado da Play Store devido à ausência de uma politica de privacidade. Isso deu início a um longo hiato no desenvolvimento que só foi retomado no início de 2020. 
    
    A estratégia de monetização usada foi a de mostrar propaganda a partir dos serviço AdMob da Goggle.
    No tempo que o aplicativo ficou disponível para download ($5$ meses), somente com publicidade exibida durante as partidas, foi possível angariar 14 dólares. Não foi possível resgatar o valor, uma vez que só se pode fazer isso em pacotes de 100 dólares.

    Considerando que o aplicativo só ficou disponível por 5 meses e que estava em seu estado base, sem qualquer atualização ou melhoramento, considera-se este retorno um ótimo resultado. Espera-se ultrapassar essa marca com o novo lançamento em 2021.

\section{Sobre o papel do desenvolvimento de jogos na equipe}

    Este é um assunto de grande complexidade. 
    
    Na ciência da computação, o assunto é tratado como um dos mais difíceis, uma vez que, assim como na criação de foguetes, é uma prática extremamente multidisciplinar. 
    Dentre as competências que se espera de uma equipe que se propõe a tal papel temos todo tipo de habilidade que oscila entre técnico e artístico de forma dramática. 

    Tendo a diversidade do assunto em vista, observa-se uma oportunidade de uma rica interlocução de agentes das mais diversas áreas, assim como vemos de um ponto de vista geral, dentro da equipe de propulsão e tecnologia aeroespacial. 
    Desse forma espera-se que a subárea de desenvolvimento de jogos usufrua dessa diversidade que existe dentro da equipe, e até a expanda ainda mais.

    Outro aspecto importante desta prática é a forma como se enxerga a programação. 
    Na pesquisa e na engenharia também é constante a prática de programar, mas isso é feito de forma pragmática e apressada.
    Esse é o natural, uma vez que o código não é o produto final, mas um meio que se precisa percorrer, seja para uma pesquisa ou para desenvolver uma estrutura ou peça para o foguete.
    Tudo isso é familiar aos outros integrantes da equipe e para qualquer um da área de engenharia.
    Mas, dentro da EPTA Entertainment a perspectiva é outra.
    O código que está sendo desenvolvido é o produto final, e como tal, deve ser feito de forma bem diferente do código acadêmico. 
    Além de funcionar ele deve ser legível e obedecer a padrões de projeto. 
    Além disso, no lugar de exercer uma funcionalidade única e específica, códigos como os desenvolvidos na subárea muitas vezes precisam ser flexíveis e robustos de forma a trabalharem harmonicamente no ecossistema lógico do aplicativo.
    
    Desenvolver essa outra perspectiva de desenvolvimento de software é extremamente positivo a qualquer engenheiro, uma vez que mais e mais o engenheiro moderno se vê cercado por programação, em qualquer especialidade.

\section{O Mínimo produto viável(MVP)}

    O jogo para celular "Epta space program" é um programa desenvolvido para Android cujo público alvo são os entusiastas da indústria aeroespacial. Desse forma, o objetivo da equipe é criar algo acessível e informativo.

\chapter{Estado do aplicativo}
    
    O aplicativo já se encontra em um estado avançado de desenvolvimento, consequência dos anos de trabalho que se foram.
    Nesse capítulo será descrito com ele se encontra no início de 2021. 
    
    Dentre as coisas descritas temos os seguintes tópicos:

    \begin{description}
        \item A certa altura (a determinar), o player tem a opção de de ir para a lua.
    \end{description}



\chapter{Plano de gestão para 2021}

\chapter{Planos futuros}

%----------------------------------------------------------------------------------------
% Bibliografia.
%----------------------------------------------------------------------------------------


% \bibliographystyle{unsrt}
% \bibliography{bibfile}



\end{document}
